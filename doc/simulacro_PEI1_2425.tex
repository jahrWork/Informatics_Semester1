\documentclass[11pt,a4paper]{article}
\usepackage[utf8]{inputenc}
\usepackage{amsmath}
\usepackage{amsfonts}
\usepackage{amssymb}




\topmargin -1cm 
\oddsidemargin 0cm
\evensidemargin 0cm

\textwidth 17 cm
\textheight 22.5 cm


\pagestyle{empty}
\begin{document}
	

\begin{center} 
	{\bf \Large Simulacro PEI1 de Informática (15/10/2023)  }
\end{center} 	
	
Calcular el parámetro $p_1$ utilizando la siguiente expresión:
\begin{equation*}
p_1 = \frac{d_1 - d_2}{d_1 + d_2} + 2,
\end{equation*}
\noindent donde $d_1$ y $d_2$ son las dos primeras cifras de tu documento de identidad. Por ejemplo, si tengo el DNI 73946724, $p_1 = (7 - 3)/(7 + 3) + 2$. En caso de no tener DNI español, usa los dígitos de tu documento de identificación.

Calcular el parámetro $p_2$ utilizando la siguiente expresión:
\begin{equation*}
p_2 = \frac{d_3 - d_4}{d_3 + d_4} + 2,
\end{equation*}
donde $d_3$ y $d_4$ son la tercera y cuarta cifra de tu documento de identidad. Para el ejemplo anterior, $p_2 = (9 - 4)/(9 + 4) + 2$.
En Moodle se debe incluir la respuesta numérica de cada pregunta así como el código
Python que ejecutado produce esa misma respuesta. Si no se incluye el código que justifique la contestación, la puntuación de la pregunta será nula.


\begin{enumerate} 
\item Escribe un programa que calcule el valor de la siguiente expresión matemática:

\begin{equation*}
\left( 1 + \frac{ p_1 - 1 }{2} p_2^2 \right)^\frac{ p_1 - 1 }{p_1}
\end{equation*}


\item Escribe un programa que calcule el valor de la siguiente expresión matemática para el valor dado de $x = 2 (p_1 + p_2) + 1$:

\begin{equation*}
\frac{\sqrt{p_1 + p_2} + x^3}{\sin(\pi x) + \exp(3 x^2)} + x
\end{equation*}


\item Escribe un programa que calcule la suma de las raíces de la ecuación de segundo grado 
$$
p_1 x^2 + (p_1 + 1)x - p_2 = 0.
$$


\item Escribe un programa que calcule las raíces de la ecuación de segundo grado 
$$
\frac{p_1}{10000} x^2 + (p_1 + 1)x - \frac{p_2}{10000} = 0,
$$ 
e inserte las raíces calculadas en la ecuación original para calcular el error (residuo) debido a la precisión numérica finita de la máquina. Devolver el mínimo error absoluto entre las dos soluciones posibles.


\item Escribe un programa que exprese la parte entera de $1000 p_1 + 500 p_2$ en base hexadecimal (base 16) como una cadena de caracteres. Ejemplo: la parte entera de $1467.654$ es $1467$ y su expresión hexadecimal es 5BB.


\item Escribe un programa que convierta el siguiente número $b = 1101010101$ expresado en base 2 a base decimal y calcule el resultado de la siguiente operación: $p_1 b + p_2$.


\item Escribe un programa que calcule la suma de todos los números primos mayores que 2 y menores que la parte entera de $20 p_1 + 10p_2$.


\item Escribe un programa que calcule el primer número primo mayor a $2p$, donde $p$ es la parte entera de $10 p_1 + 20 p_2$.


\item Escribe un programa que calcule la suma de los $N$ primeros términos de la siguiente serie numérica:
\begin{equation*}
S_N = 1 + \sum_{n=1}^N \frac{1}{n!},
\end{equation*}
siendo $N$ la parte entera de $10(p_1 + p_2)$.

\item  Dada la siguiente serie numérica:
\begin{equation*}
S = \sum_{n = 1}^\infty \frac{1}{n^2} = \frac{\pi^2}{6},
\end{equation*}
escribir un programa que calcule el número de términos a sumar para que el error absoluto en el cálculo aproximado de la serie sea menor que $p_1 \times 10^{-4}$, es decir, determinar $N$ tal que $|S_N - \frac{\pi^2}{6}| < p_1 \times 10^{-4}$, donde $S_N$ es el valor de la suma de los $N$ primeros términos de la serie.

\end{enumerate} 

\end{document}