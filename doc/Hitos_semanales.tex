%
\documentclass[12pt,spanish]{article}
   


\usepackage[spanish]{babel}
\usepackage[utf8]{inputenc}

\usepackage[fleqn]{amsmath}
\usepackage{nccmath}
\usepackage{amsmath}
\usepackage{nccmath}

\usepackage{latexsym}  %\mathbb{R}

%**************

\usepackage{graphicx}
\usepackage{bezier}
\usepackage[spanish]{babel}
\usepackage{makeidx}
\usepackage{latexsym}
% acentos y cosas varias
\usepackage{epstopdf}
\usepackage{amssymb}
\usepackage{curves}
\usepackage{rotating}
\usepackage{amsmath}
\usepackage{epic}


%\usepackage{HVDASHLN}
%\usepackage{epsfig}


\usepackage[utf8]{inputenc}
\usepackage[T1]{fontenc}
\usepackage{lmodern} % load a font with all the characters


\usepackage{hyperref} 
\usepackage{xcolor}

%**************



%*************************************************************************************************************
%***  Formato del documento                                                                                ***
%*************************************************************************************************************

\textwidth       16 cm
\textheight      24 cm

\oddsidemargin    0 cm
\evensidemargin   0 cm

\topmargin        -1 cm


\pagestyle{empty}

%\parskip = 1.5 \baselineskip


%*************************************************************************************************************
%***  Comandos para distintos conjuntos de numeros                                                         ***
%*************************************************************************************************************

\newcommand{\RR}{{\sf I}\hspace*{-1.5pt}{\sf R}}
\newcommand{\NN}{{\rm I}\hspace*{-1.5pt}{\rm N}}
\newcommand{\PP}{{\sf I}\hspace*{-1.5pt}{\sf P}}
\newcommand{\QQ}{\hspace*{2pt}{\sf I}\hspace*{-5pt}{\sf Q}}
\newcommand{\ZZ}{{\sf Z}\hspace*{-4.5pt}{\sf Z}}
\newcommand{\CC}{\hspace*{2.5pt}{\sf I}\hspace*{-5pt}{\rm C}}


%*************************************************************************************************************
%***  Comandos para subindices y superindices                                                              ***
%*************************************************************************************************************
 
\newcommand{\Ri}{_{_R}}  
\newcommand{\Le}{_{_L}}

      

%*************************************************************************************************************
%***  Inicio de documento                                                                                  ***
%*************************************************************************************************************



\begin{document}

\begin{center}
	{\bf Asignatura : Informática}                                               
	\\
	{\bf Grado en Ingeniería Aeroespacial - ETSIAE}                                            \\

\end{center}

\vspace{0.5cm}

\noindent
{\bf \Large Hitos semanales del primer semestre: 
}                                                                               
     \\

\vspace{-1.cm}

\begin{itemize}
	
	
\vspace{1cm}
\item {\bf Hito 1 :} Primeros pasos con expresiones, asignaciones y sentencias. 
\begin{enumerate} 
\item Escribir un programa  que abra una pantalla que imprima ``Hello world''. 
\item Crear variables de tipo entero, real, complejo, booleanas y
cadenas de caracteres. Verificar su tipo mostrándolo por pantalla. 
\item Escribir un programa  que calcule la la nota media del primer 
semestre de informática basada en las notas de la pruebas de evaluación intermedia: PEI1, PEI2 y PEI3
y de acuerdo a una ponderación establecida. Exigir el criterio  PEIi $ \ge 3$ para poder hacer media. 
\item Evaluar la expresión 
$$ \frac{ 3 x^3 + 5 x - 1 }{ e^x + 3 \sin x - \log(x) }
$$
para los valores $ x = 1 $ y $ x=3$.
\end{enumerate}
%Escribir el program en un 
%compilador online: 
%
%\color{red} 
%\begin{center} 
%       \href{https://www.mycompiler.io/new/fortran}{Compilador online}
% \end{center} 
%\color{black} 
               
%Instalar el entorno Microsoft Visual Studio  
%mediante: 
%  
%  \color{red} 
% \begin{center} 
%        \href{https://www.amazon.es/s?k=hernandez+rapado+moreno+visual&__mk_es_ES=AMAZON&ref=nb_sb_noss}
%        {Libro en amazon.es}
%        
%        \href{https://github.com/jahrWork/Visual-Studio-projects/blob/master/doc/Manual_Visual_Studio_dic_2021.pdf}
%             {PDF del libro} 
%  \end{center} 
%\color{black}   
%
%La implementación de todos los hitos se encuentran en el enlace del github : 
% \color{red} 
% \begin{center} 
%        \href{https://github.com/jahrWork/Informatics_S1_ETSIAE}{GitHub: Juan A. Hernández Ramos} 
%  \end{center} 
%\color{black}   


%******************************************************
\vspace{1cm}
\item {\bf Hito 2 :}    Raíces de una ecuación algebraica de segundo grado. 

Calcular las raíces de una ecuación algebraica de segundo grado
de coeficientes reales de la forma: 
$$
a x^2 + b x + c = 0.
\label{ecuacion_segundo_grado}
$$
Para todo  $a, b, c \in \mathbb{R}$.
El programa debe contemplar la posibilidad de que cualquiera de los coeficientes
sea cero. Los resultados deben aparecer por pantalla. Discutir el 
condicionamiento numérico de las soluciones en función de $a, b, c $. 
 
%******************************************************
\vspace{1cm}
\item {\bf Hito 3 :}    Números primos y números perfectos.  
\begin{enumerate} 
\item Determinar si un número $ N $ es primo. 
\item Determinar si un número $ N $ es perfecto. 
\item Calcular los N primeros números primos. 
\item Calcular los N primeros números perfectos. 
\end{enumerate}


%******************************************************
\vspace{1cm}
\item {\bf Hito 4 :}  
Números enteros en bases diferentes. 
\begin{enumerate} 
\item Determinar las unidades, centenas y millares de un número entero $ N $. 
\item Dado un número entero $ N $ obtener su representación en base 2. 
\item Dado un número entero $ N $ obtener su representación en base 16. 
\end{enumerate}
 
 
\newpage 
%***************************************************   
\item {\bf Hito 5 :}   Suma de series numéricas finitas e infinitas,    
$$
   S_N = \sum_{n=1} ^{N} \ a_n, \qquad 
   S =  \sum_{n=1} ^{\infty} \ a_n.
$$ 
donde $ a_n $ representa  los términos de una sucesión numérica $ a_n: \mathbb{N} \rightarrow\mathbb{R}$,
$ S_N $ la suma de los $ N $ primeros términos de la sucesión y $ S $ la suma de los infinitos términos. 
                                                                         
\begin{enumerate}    
	\item Calcular la suma de los primeros $ N $  números naturales.	
	\item Calcular la suma de los $ N $  primeros números naturales impares.
	
	\item Calcular el factorial de un número $ N $. 
	Implementar este cálculo de dos
	formas distintas : incrementos positivos e incrementos negativos
	 de un  bucle. 
 %   Comprobar el máximo valor de $N$ válido para diferentes 
 %tipos de dato.
    El número $ N $ se debe poder introducir por teclado y el resultado aparecerá
	por pantalla. 
	\item Calcular 
	 las siguientes sumas de los $ N $ términos de las series  numéricas: 	
	$$
	S_{1N} = \sum _{n=1} ^{N}  \  \frac{ 1 }{  2^n  }, \qquad 
	S_{2N} = \sum _{n=1} ^{N}  \  \frac{ 1 }{  n^2  }, \qquad  
	S_{3N} = \sum _{n=1} ^{N}  \  \frac{  (-1)^{n+1} }{  2^n  }, \qquad
	S_{4N} =\sum _{n=1} ^{N} \ \frac{ 1 } { n ! }. \qquad  
	$$
	\item Calcular 
		 las siguientes sumas de las series numéricas: 	
		$$
		S_{1} = \sum _{n=1} ^{\infty}  \  \frac{ 1 }{  2^n  }, \qquad 
		S_{2} = \sum _{n=1} ^{\infty}  \  \frac{ 1 }{  n^2  }, \qquad  
		S_{3} = \sum _{n=1} ^{\infty}  \  \frac{  (-1)^{n+1} }{  2^n  }, \qquad
		S_{4} =\sum _{n=1} ^{\infty} \ \frac{ 1 } { n ! }. \qquad  
		$$
	\item Comparar los resultados obtenidos $ S_{iN} $ y $ S $. 
	\item Dado $ \epsilon $ determinar el valor de $ N $ que hace $ | S_i - S_{iN} | < \epsilon $. 
	\item Explicar las diferencias entre los diferentes valores de $ N $.
	\item Reescribir todos los apartados con paradigma declarativo y verificar los resultados. 
	
\end{enumerate}         

\newpage
%***********************************************
\item {\bf Hito 6 :}   Vectores y matrices. 

Sean los vectores $V, W \in \mathbb{R}^N$ y la matriz de Vandermode $A \in 
{\cal M}_{N \times N} (\mathbb{R})$: 
$$
\{ v_i =  1/i^2, \ \  i=1, \ldots N \} \qquad  \{  w_i = (-1)^{i+1}/(2i+1), \ \   i=1, \ldots N  \} 
$$
$$
\{ a_{ij} = (i/N)^{j-1}, \ \ i=1, \ldots N, \ \  j=1, \ldots N \}. 
$$
Escribir un programa que realice las operaciones siguientes y que muestre los
resultados por pantalla: 
\begin{enumerate}
	\item Sumar todas las componentes del vector $V$. 
	\item Sumar todas las componentes de la matriz $A$.   
	\item Sumar las componentes del vector $V$ mayores que cero.
	\item Sumar las componentes de la matriz $A$ menores que cero.   
	\item Producto escalar de los vectores $V$ y $W$.   
	\item Producto escalar del vector $V$ y la columna $N$ de la matriz $A$. 
	\item Multiplicar la matriz $A$ por el vector $V$.
	\item Matriz traspuesta de la matriz $A$   
	\item Obtener el valor máximo de la matriz $A$ y su posición.
\end{enumerate}

Verificar los resultados con las funciones intrínsecas correspondientes. 


%***********************************************
\item {\bf Hito 7 :}   Asignación dinámica de memoria.

Dada la matriz  de Vandermonde $A_M \in {\cal M}_{M \times M} (\mathbb{R})$ del hito anterior,
calcular las siguientes operaciones: 


\begin{enumerate}	
	\item Suma de las trazas de $ A_M $ para matrices de diferente dimensión. 
	$$  S_1 = \sum_{M=1} ^{10} trace(A_M) $$ 
	\item  Suma de las trazas del cuadrado de $ A_M $ para matrices de diferente dimensión. 
	$$  S_2 = \sum_{M=1} ^{5} trace(A_M^2) $$ 
	\item Traza de la suma de potencias de $ A_M $  con $ M=8 $.
	$$  S_3 = trace \left( \sum_{k=0} ^{5} A_8^k  \right) $$
\end{enumerate}  
 
Implementar la operaciones anteriores mediante un módulo llamado \textbf{Algebra} 
que contenga la función  \textbf{trace} que calcula la traza de una matriz, la 
función  \textbf{Vandermonde} que crea la matriz  $ A_M \in { \cal M } _{ M 
\times M } ( \mathbb{R} )$ 
 y la función \textbf{power} que permite calcular la potencia k--ésima de 
 cualquier matriz cuadrada. 

	
	  
	



\newpage

%***********************************************
\item {\bf Hito 8 :}  Primitivas y derivadas numéricas de funciones.  Funciones 
definidas a trozos. Impelentar la función \texttt{integral} que calcule la 
integral de una función genérica $ f: \mathbb{R} \rightarrow \mathbb{R} $ entre 
$ [a, b] $ y la 
función  \texttt{derivative} que calcule la 
derivada de una función genérica $ f: \mathbb{R} \rightarrow \mathbb{R} $ en $ 
x = x_0 $ 
mediante las siguientes fórmulas aproximadas: 
$$
 \int _a ^b f(x)  dx \approx \sum_{k=0} ^N f(x_k) \Delta x, \qquad 
 \frac{df}{dx} ( x_0  ) \approx \frac{ f(x_0+h) - f(x_0) }{ h }, 
$$
donde  $  \Delta x = ( b- a) /N  $ y $ h $ es un valor pequeño. 
\begin{enumerate}


\item Validar las funciones \texttt{integral} y \texttt{derivative} mediante 
funciones simples con derivada e integral analítica.  

\item Implementar la subrutina \texttt{plot} que dibuje la función genérica $  
f: \mathbb{R} \rightarrow \mathbb{R} $  desde $ x = a $ hasta a $ x =b$. 
Esta subrutina deberá calcular la imagen de la función anterior en $ M +1 $ 
puntos de acuerdo a la siguiente partición equiespaciada: 
$$ \{ x_i = a + i \Delta x, \ \ i = 0 \ldots  M \}, \qquad 
\displaystyle \Delta x = \frac{b-a}{M}. $$

\item Representar la función seno para $ x \in [ -2 \pi, +2 \pi]$.  
\item Implementar la función $ f(x)$ definida a trozos siguiente:  
$$
f(x) =
\left\{
\begin{array}{ll}
\displaystyle                          
0 , \hspace{1cm} \hspace{2cm} &  \displaystyle  -\infty \leq x \leq - 
\frac{\pi}{2},      \\ \\
\displaystyle                                                            
\cos{( x)} ,  & \displaystyle - \frac{\pi}{2} < x <  \frac{\pi}{2}, \\ \\
\displaystyle                                                            
0 ,              & \displaystyle  \frac{\pi}{2} \leq x \leq 
+\infty.                                                                        
                        
\end{array}                                                                   
\right.                                                     
$$ 


\item Representar  la función anterior definida a trozos  para $ x \in [ -2 
\pi, +2 \pi]$. 

 
\item Representar la integral y la derivada de la función definida a trozos 
anterior para $ x \in [ -2 \pi, +2 \pi]$.  
	
\end{enumerate}

Las funciones \texttt{integral}, \texttt{derivative} y \texttt{plot} deberán 
estar en un módulo llamado \texttt{calculus}      


\newpage 
%***********************************************
\item {\bf Hito 9 :}  Series de potencias  

Aproximar  mediante un desarrollo truncado en serie de Taylor de la forma:
\[  f(x) = \sum_{k=0} ^M a_k \  (x-x_0)^k, \qquad \qquad a_k = \frac{  f^{(k)} 
(x_0)  }{ k! },  \]  
para las funciones $f : \mathbb{R} \rightarrow \mathbb{R}$, siguientes:    
\begin{enumerate}
	\item $f(x) = e^x$,     
	\item $f(x) = \sin(x)$,   
	\item $f(x) = \cosh(x)$, 
	\item $f(x) = \displaystyle \frac{1}{1 - x}$.               
\end{enumerate}
Para cada desarrollo en serie se pide:    



\begin{enumerate}
	\item Calcular el desarrollo para un valor de $M$ genérico ($M=1, 2, 3, 4, 5, \hdots$) mediante un bucle desde $ k=0$ 
	hasta $ k = M $. 
	\item Implementar un desarrollo para infinitos términos que para de sumar cuando se alcance la máxima precisión. 
	Realizar la implementación mediante 
	una sentencia de control  \verb|while|.   
	\item Comparar con las funciones intrínsecas correspondientes.     
	% \item  Dibujar la función a aproximar junto con las param�tricas que constituyen los desarrollos anteriores con $ M=1, 2, 3, 4, 5. $
	\item Discutir los resultados obtenidos.
\end{enumerate}



\newpage 

%***********************************************
\item {\bf Hito 10 :}   Sentencias de entrada y salida.

\vspace{0.5cm}    

Crear los ficheros de datos ForTran con nombres  \verb|data1.txt| e 
\verb|data2.txt| con la información siguiente:

\vspace{1cm}

Contenido del fichero de entrada \verb|data1.txt| :

\begin{verbatim}
VARIABLES=t, x, y, z, u
1.2,      3.4,     6.2,    -14.0,    0.1
-25.2,    -8.6,    5.1,     9.9,    17.0
-1.0,     -2.0,    -5.4,    -8.6,    0.0
3.14,     -11.9,   -7.0,    -12.1,   9.2
6.66,    5.32,    0.001,   0.2,     0.001
\end{verbatim}

\vspace{1cm}

Contenido del fichero de entrada \verb|data2.txt| :

\begin{verbatim}
VARIABLES=t, x1, x2, x3, x4, x5, x6
1.2,      3.4,     6.2,    -14.0,    0.1,      4.89,   7.54
4-25.2,    -8.6,    5.1,    12.0,     9.9,      12.24,  17.0
0.0,      34.5,    -1.0,   -2.0,     -43.04,   -8.6,    0.0
3.14,     -11.9,   71.0,   7.0,     17.0,      -12.1,   9.2
 6.66,     5.32,    0.001,   0.2,     0.001,    0.008,   -0.027
54.0,     77.1,    -9.002,  -13.2,   0.017,    65.53,   -0.021
 23.04,    -51.98,  -34.2,   9.99,    5.34,     8.87,    3.22 
\end{verbatim}

\vspace{0.5cm}


Escribir una función \verb|load_matrix| que permita crear una matriz A con las dimensiones 
del fichero de datos y que almacene los valores de dicho fichero mediante 
 asignaciones del tipo: 
 
 \hspace{2cm} \verb|A = load_matrix('data1.txt')| 
 
 \hspace{2cm}  \verb|A = load_matrix('data2.txt')|

La función   \verb|load_matrix| deberá permitir la carga de datos con cualquier formato. 
Una vez cargados los valores de la matriz $ A $, crear dos vectores $ U$ y $V$  que contengan la primera y segunda columnas de la matriz A. 
     

	
\end{itemize}
%
\end{document} 	
	
	
	
	
	
%	
%	
%\newpage 
%
%
%\begin{center} 
%{ \LARGE  \bf Segundo Semestre }
%\end{center} 	
%
%\vspace{1cm} 
%	
%	
%%******************************************************
%    \item {\bf Hito 1 :}   Sistemas lineales de ecuaciones.
%
%Implementar un módulo para la resolución de sistemas lineales de ecuaciones algebraicas.
%Los métodos de resolución propuestos son el de eliminación Gaussiana, factorización LU,
%factorización LU de la biblioteca {\it Numerical Recipes} y Jacobi.
%
%Para cada método se pide:
%\begin{itemize}
%     \item Validar los resultados con varios casos de prueba con dimensiones distintas.
%     \item Evaluar tiempos de ejecución.
%     \item Comparar resultados con los métodos restantes.          
%\end{itemize}
%
%Aplicación : Estudiar el condicionamiento de sistemas lineales de ecuaciones para matrices
%aleatorias y de Vandermonde. 
%
%
%\vspace{0.5cm}
%
%
%\item {\bf Hito 2 :} Autovalores y autovectores. 
%
%Implementar un módulo para el cálculo de autovalores y autovectores de una matriz.
%Los métodos de resolución propuestos son el método de la potencia y el método de la potencia inversa.
%Implementar el método de la potencia inversa a partir de la matriz inversa y resolviendo el sistema lineal
%correspondiente. 
%
%Para cada método se pide:
%\begin{itemize}
%	\item Validar los resultados con varios casos de prueba con dimensiones distintas.
%	\item Evaluar tiempos de ejecución. Comparar tiempos de ejecución del método de la potencia inversa
%	      mediante los dos algoritmos propuestos : matriz inversa y solución del sistema lineal.       
%\end{itemize}
%
%Aplicación : Estudiar el condicionamiento de sistemas lineales de ecuaciones para matrices
%aleatorias y de Vandermonde. Calcular la relación $\lambda_{max} / \lambda_{min}$ de los casos de prueba
%presentados en el hito 1 y relacionar y discutir los resultados. 
%     
%
%%********************************************************************************************************************
%\newpage
%
%    \item {\bf Hito 3:} Derivación numérica. 
%    
%    \begin{enumerate}   
%    	
%    	\item Obtener las fórmulas de las derivadas numéricas primeras descentradas, con tres puntos
%    	equiespaciados a una distancia $\Delta x $.
%    	
%    	\item A partir de la función $f(x) = \mbox{e}^{x}$ en el punto $x=0$, 
%    	representar gráficamente el error total de las derivadas numéricas 
%    	frente al valor de $\Delta x$ en precisión simple y doble.
%    	En particular, representar gráficamente las derivadas primeras adelantada (definición de derivada), centrada y descentradas y la 
%derivada segunda,
%    	con tres puntos equiespaciados a una distancia $\Delta x $. 
%    	Discutir los resultados obtenidos.
%    	
%    	
%    	
%    	\item Resolver los problemas de contorno en ecuaciones diferenciales ordinarias siguientes:
%    	
%    	\begin{itemize}
%    		
%    		\item {\bf Problema 1:}	         
%    		$$
%    		u^{\prime\prime} + u = 0,\quad
%    		x\in[-1,1], \qquad \qquad u(-1)=1,\quad u(1)=0,
%    		$$
%    		
%    		\item {\bf Problema 2:}	    
%    		$$
%    		u^{\prime\prime} + u^{\prime} - u=\sin(2 \pi x),\quad
%    		x\in[-1,1], \qquad \qquad u(-1)=0,\quad u^{\prime}(1)=0.
%    		$$
%    	\end{itemize}
%    	
%    	
%    	Para los problemas citados anteriormente se pide: 
%    	
%    	\begin{enumerate}
%    		\item A partir de las derivadas numéricas con tres puntos equiespaciados escribir el sistema de
%    		ecuaciones resultante.
%    		
%    		\item Obtener la solución numérica mediante la resolución de un sistema lineal de
%    		ecuaciones, con $N = 10$ y $N = 100$. 
%    		
%    		\item Representar gráficamente los resultados obtenidos.
%    	\end{enumerate}
%    	
%    \end{enumerate}
%    
%\vspace{0.5cm}
%    
%%******************************************************
%\item {\bf Hito 4 :} Integración numérica.
%
%Implementar un módulo para la resolución numérica de integrales definidas de funciones $F:\mathbb{R} \rightarrow \mathbb{R}$.
%Los métodos de resolución propuestos son las reglas del rectángulo, punto medio, trapecio y Simpson.
%Implementar un módulo de funciones $F:\mathbb{R} \rightarrow \mathbb{R}$ de prueba para validar los métodos numéricos
%propuestos. Este módulo debe contener al menos tres funciones con funciones primitivas conocidas y una función cuya 
%función primitiva sea desconocida.
%
%Evaluar el error de las soluciones numéricas para cada método propuesto y para distintos valores del incremento de la
%partición.   
%
%\newpage
%           
%%******************************************************
%\item {\bf Hito 5 :} Ecuaciones no lineales.
%
%Implementar un módulo para la resolución numérica de ecuaciones no lineales.
%Para funciones  $F:\mathbb{R} \rightarrow \mathbb{R}$, los métodos de resolución propuestos son el de la bisección y el de 
%Newton-Raphson.   
%Para funciones  $F:\mathbb{R}^N \rightarrow \mathbb{R}^N$, se proponen el método de Newton-Raphson con matriz Jacobiana analítica y 
%el método de Newton-Raphson con matriz Jacobiana numérica.
%Para la validación de los métodos propuestos, se pide implementar un módulo con al menos tres funciones $F:\mathbb{R} \rightarrow \mathbb{R}$
%y al menos tres funciones $F:\mathbb{R}^N \rightarrow \mathbb{R}^N$. Este módulo debe contener las derivadas y matrices Jacobianas 
%correspondientes de las funciones propuestas.  
%
%En el informe correspondiente, presentar tablas de soluciones numéricas en cada paso de iteración para las funciones de prueba propuestas.
%
%    
%    
%
%
%
%    
%
%
%
%
% 
%\end{itemize}
%
%\end{document} 
% 
% 
% 
% 
% 
% 
% 
% 
% 
% 
% 
% 
% 
% 
% 
% 
% 
% 
% 
% 
% 
% 
% 
% 
% 
% 
% 
% 
% 
% 
% 
% 
% 
% 
% 
% 
%\newpage
%
%
%%***********************************************
%    \item {\bf Hito 5 : }   Funciones y gráficas. 
% 
% \begin{enumerate}    
% \item Escribir un programa que dibuje las siguientes funciones: 
% 
% 
% 
% \begin{fleqn}
%  \[ f(x)  = \sqrt{ 1 - ( |x| -1 )^2 } \]
%  
%  \[ f(x) = \arccos(1 - | x | )  + \pi \]
%  \[ f(x) =  \frac{ \cos( 70 x ) }{ (1 + x^2 )^2  }\]
% \[ f(x) = \sqrt{ \cos(x) }  \]
%  \[ f(x) =  8 x \sin( x^2 + 3)  \]
% \[ f(x) = \prod_{k=0} ^M ( x - k/M )  \]
% \end{fleqn}
% 
%\item Dibujar el lugar geométrico de las raíces de: 
%
%\[ x^2 + 2 \zeta \omega_0 x + \omega_0 ^2 = 0,\]
%
%con $ \omega_0 = 1 $ para todo $ \zeta \ge 0$. 
% 
%\end{enumerate}
%
%
%
%%***********************************************
%    \item {\bf Hito 6 :}  Aproximar  mediante un desarrollo en serie de potencias de la forma
%    \[  f(x) = \sum_{k=0} ^M a_k \  x^k, \qquad a_k = \frac{  f^{(k)} (0)  }{ k! }  \]  
%    las  siguientes funciones: 
%   
%    \begin{fleqn}
%  \[   f(x)  = \cos(x)  \]
%    \[ f(x) = \cosh(x) \]
% \end{fleqn}
%    
% 
% 
% 
% \begin{enumerate}
%   \item Calcular los desarrollos anteriores para un valor de $ M $ genérico.  
%   \item  Dibujar la función a aproximar junto con las parmétricas que constituyen los desarrollos anteriores con $ M=1, 2, 3, 4, 5. $
%   \item Discutir los resultados obtenidos.
% \end{enumerate}
% 
% 
%
%%***********************************************
%    \item {\bf Hito 7 :}   Ceros de funciones. 
%    
%  Dada una función genérica $ f(x) $ escribir una programa que permita calcular los ceros mediante: 
% 
% 
% \begin{enumerate}
%   \item Algoritmo de la bisección. 
%   \item Algoritmo de Newton. 
%   \item Hacer aplicación a una función ejemplo y dibujar el resultado obtenido. 
% \end{enumerate}
% 
%\end{itemize}

\end{document}        
    
   
 
