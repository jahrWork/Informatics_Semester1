\documentclass[12pt]{article}

\usepackage[utf8]{inputenc}
%\usepackage[T1]{fontenc}
\usepackage{amsmath}
\usepackage{amsfonts}
\usepackage{amssymb}
%\usepackage[version=4]{mhchem}
\usepackage{stmaryrd}
\usepackage{bbold}
\usepackage{hyperref}
%\hypersetup{colorlinks=true, linkcolor=blue, filecolor=magenta, urlcolor=cyan,}
\urlstyle{same}


\topmargin -1cm 
\oddsidemargin 0cm
\evensidemargin 0cm

\textwidth 17 cm
\textheight 22.5 cm




\begin{document}
	
\begin{center} 
	{\bf \Large Ejercicios propuestos para las pruebas: \\ \vspace{0.5cm} PEI1 y PEI2 de Informática  }
\end{center} 

\vspace{1cm}
\begin{enumerate} 	
	
\item Escribe un programa que calcule la siguiente expresión matemática 
para un valor dado de $ x $: 
$$
    \frac{ 3 x^3 + 1}{   \cos(x^2) + 5 \ log|x| }
$$
	
\item Escribe un programa que dados tres números reales: $a$, $b$ y $c $ $ (a \neq 0)$ 
y calcule el módulo máximo de las raíces de la ecuación de segundo grado:
$$
a x^{2}+b x+c=0
$$


\item  Escribe un programa que calcule la suma de los $N$ primeros números primos.

\item  Escribe un programa que calcule la suma de los $ N $ primeros   números perfectos.

\item Escribe un programa que permita obtener la representación de un entero en base dos. 
El resultado debe ser una cadena de caracteres. 

\item Escribe un programa que permita obtener la representación de un entero en base dieciséis. 
El resultado debe ser una cadena de caracteres. 

\item Escribe un programa para calcular la suma de los $ N $ primeros términos de la  serie numérica 
$$
   S_N = \sum_{n=1} ^{N} \frac{1} { n! } 
$$

\item Escribe un programa para calcular la suma de infinitos términos de la siguiente serie numérica: 
$$
S = \sum_{n=1} ^{\infty} \frac{1} { n^2 } 
$$
con precisión $ \epsilon $. Sabiendo que el resultado es $ \pi^2 / 6 $, sumar mientras $ S - S_N > \epsilon $ 
y calcular el número mínimo de términos que hay que sumar para alcanzar esa aproximación. 

\item  Escribe una función que reciba un número natural $N$ y devuelva una lista con los números 
primos menores o iguales que el dado.
Nota: Para $N<2$ devolverá una lista vacía.


\item  Escribe una función que dado un número entero $N$, devuelva una lista con los primeros $N$ números perfectos.
Nota: Un numero es perfecto si es la suma de todos sus divisoresde sin 
contar el mismo coincide con dicho número.
Ej 6 es un número perfecto $6=1+2+3$; $28$ es un número perfecto $28=1+2+4+7+14$.
		










\newpage 

\item Escribe una función que reciba dos listas y que devuelva  dos listas: una con los elementos 
comunes y otra con los elementos diferentes.

	
\item Escribe una función  que reciba dos cadenas de caracteres y que devuelva una lista 
con las letras que aparecen en ambas cadenas.
Por ejemplo:
\begin{itemize}
		\item Para 'melón' y 'tomate' devolverá ['m', 'e'] (Distingue entre vocales con o sin tilde).
		\item Para 'Violín' y 'lava' devolverá ['v', 'l'] (No distingue entre mayúsculas y minúsculas).
\end{itemize}

		
\item Escribe una función que elimine el elemento segundo mas pequeño de una lista. 
La función deberá devolver una lista con el elemento eliminado. 
	


\item Escribe una función que cuente los elementos de una lista de números 
que están entre $ [a, b] $ siendo $a$ y $b$ dos números reales. 
La función devolverá el número de elementos de la lista que cumplan esa condición. 


\item Escribe una función que determine si una cadena de caracteres es un palíndromo.
Un palíndromo  es una palabra o frase que se lee igual en un sentido que en otro.
Ej.  "somos o no somos" es un palíndromo. 
\item Escribe una función que determine si una cadena de caracteres es un palíndromo.
Completa la función anterior teniendo en cuenta que los caracteres en mayúscula y las vocales acentuadas 
se consideran iguales para determinar si es palíndromo. 
Ej:  "Dábale arroz a la zorra el abad" es un palíndromo. 
\item Escribe una función que cuente el número de vocales que aparecen en una cadena de caracteres. 

\item Escribe una función que elimine los elementos duplicados de una lista. 

\item Escribir una función de nombre: \texttt{divisors} de devuelva una lista de los divisores de un número $N$.  

\item Escribir una función de nombre: \texttt{mcd} que calcule el máximo común divisor de los números introducidos
mediante el uso de la función anterior. 



\newpage 


\item Dado el siguiente texto: 
\begin{verbatim}
	Hit the road Jack and don't you come back
	No more, no more, no more, no more
	Hit the road Jack and don't you come back no more
	What you say?
	Hit the road Jack and don't you come back
	No more, no more, no more, no more
	Hit the road Jack and don't you come back no more
	Old woman, old woman, don't treat me so mean
	You're the meanest old woman that I've ever seen
	I guess if you said so
	I'll have to pack my things and go (that's right)
	Hit the road Jack and don't you come back
	No more, no more, no more, no more	
\end{verbatim}
Determinar: 
\begin{enumerate} 
\item  El número de palabras diferentes. NOTA: las comas y los paréntesis
 no se consideran palabras. 
Tampoco son palabras los retornos de carro del final de línea.
\item  El número de repeticiones de cada palabra.
\item  La palabra con mayor número de repeticiones. 
\end{enumerate} 


\item Escribe una función que determine si una subcadena \texttt{s} pertenece a una cadena de 
caracteres \texttt{S} dada.
La función deberá devolver una lista con todos los índices de \texttt{S} donde empiece la subcadena \texttt{s}.
Ej: S = "1234561234", find s="34". Return [2, 8]



\newpage 

\item Escribe una función que calcule una partición equespaciada del 
segmento $ [a, b] $ de la recta real en $ N  $ trozos. 
Devolverá una  vector $x$ cuyas componentes serán los nodos $ x_i $ de la partición
con $ N+1 $ componentes.  




\item Crear una función que calcule las imágenes de la función $ f(x) $ 
	en los puntos de la partición. La función tendrá dos argumentos: 
	la función y una lista que contega la partición donde se calculan las imágenes. 
	Devolverá una lista con los valores las imágenes. 
	

\item Crear una función que calcule las imágenes de la función $ f(x) $ 
	en los puntos de la partición. La función tendrá dos argumentos: 
	la función y un vector que contenga la partición donde se calculan las imágenes. 
	Devolverá un vector cuyas componentes sean las imágenes. 
	


\item La expresión general de un polinomio $p(x)$ de grado $N$ es:
	$$
	p(x)=a_{0}+a_{1} x+a_{2} x^{2}+\cdots+a_{N} x^{N}
	$$
Se pide crear una función que tenga como argumentos de entrada una lista con los coeficientes $ a_i $ del polinomio
y el valor $ x $ donde se pretende evaluar. La función deberá devolver 
el valor del polinomio evaluado en el punto $x$.
La función debe ser válida para un número arbitrario de coeficientes 
y el argumento de entrada $x$ es un valor real.	

	
\item Escribir una función que calcule las imágenes  $y_i = f(x_i) $
de las componentes de un vector $ x $ mediante la función $ f(x)$ genérica. 

	

\item Hacer aplicación de la función anterior cuando 
	$$
	f(x) = \left \{
		       \begin{array}{cc}
				    \cos (\pi x) & x \leq 0 \\ 
					1 + \sin(\pi x) & x>0 
				\end{array} 
	       \right.
	$$ 


		
\item Escribe una función de nombre gaussian que reciba como argumento una 
variable real $x$, otra variable real $m$ cuyo valor por defecto sea $0$ 
otra variable real $s$ cuyo valor por defecto sea 1. Devolverá el resultado de la operación:
$$ 	
f(x) = \frac{1}{\sqrt{2 \pi} s}  \exp \left[  -\frac{1}{2}  \left(\frac{x-m}{s} \right)^{2} \right]
$$

		

	
	
\item Una partícula se mueve en el plano describiendo un movimiento 
circular de radio $R$ centrado en $(x, y)=(0,0)$. 
Escribir una función que reciba el radio de giro $\mathrm{R}$ y el instante temporal $t$ 
como variables reales $y$ y que devuelva la posición: $[x(t), y(t)]$ como una tupla real usando la fórmula:
	$$
	\begin{aligned}
		& f: \mathbb{R} \rightarrow \mathbb{R}^{2} \\
		& \vec{r}(t): t \rightarrow(R \cos (t), R \operatorname{sen}(t))
	\end{aligned}
	$$



\item Escribir una función con dos argumentos vectoriales de $\mathbb{R}^{N}$, que representan la posición 
de dos puntos, y que calcule a distancia euclídea entre dichos puntos.  
	
	
\item Escribir una función de nombre: \texttt{is\_inside(x, R, c)} que determine si un punto $ x \in \mathbb{R}^{N} $ 
está en el interior de una hiperesfera de radio $ R $ y centro $ c \in \mathbb{R}^{N}  $.
La función devolverá un valor lógico True o False dependiendo de la posición del punto.
	
\item Escribir una función de nombre derivative($x, h$, funcion), 
que recibe como argumentos dos valores reales $x$, $h$ y una función real de variable real. 
Devolverá el valor de la derivada de funcion en x de acuerdo con la fórmula:	
	$$
	f^{\prime}(x) \approx \frac{ f(x+h)-f(x) } { h }
	$$
	
	
\item Escribir una función que dado un vector $ x $, una función $ f(x)$ y la función 
\texttt{derivative} del ejercicio anterior calcule las derivadas de $ f(x) $ en las componentes de  $ x $ 

		









\item Escribe una función de nombre \texttt{Taylor\_seno( $x$, tol)} 
que aproxime 
mediante un desarrollo en serie de potencias la función \texttt{seno(x)} 
con una toleracia dada: \texttt{tol} en un punto \texttt{x} dado.  
% Devolverá la aproximación del valor del seno en $x$, 
La función deberá implementar la aproximación mediante el polinomio de Maclaurin.
	$$
 	\sin(x) \sim T(x)=\sum_{n=1}^{M} \frac{(-1)^{n+1} x^{2 n-1}}{(2 n-1) !}
		$$
El número de términos en el sumatorio vendrá determinado 
por la variable tol de manera que se cumpla: $|\sin(x)-T(x)|<$ \texttt{tol}.
	
% \item Escribe una función en Python de nombre Taylor\_coseno ( $x$, tol) 
% que reciba como argumento dos variables reales $x$, to 1. 
% Devolverá la aproximación del valor del coseno en $\mathrm{x}$, 
% usando la aproximación mediante el polinomio de Maclaurin.
% 	$$
% 	\operatorname{coseno}(x) \sim T(x)=\sum_{n=0}^{M} \frac{(-1)^{n} x^{2 n}}{(2 n) !}
% 	$$
% El número de términos en el sumatorio vendrá determinado 
% por la variable tol de manera que se cumpla: $|\operatorname{coseno}(x)-T(x)|<$ tol


\item Series de potencia.  
Aproximar  mediante un desarrollo truncado en serie de Taylor de la forma:
\[  f(x) = \sum_{k=0} ^M a_k \  (x-x_0)^k, \qquad \qquad a_k = \frac{  f^{(k)} 
(x_0)  }{ k! },  \]  
para las funciones $f : \mathbb{R} \rightarrow \mathbb{R}$, siguientes:    
\begin{enumerate}
	\item $f(x) = e^x$,     
	\item $f(x) = \sin(x)$,   
	\item $f(x) = \cosh(x)$, 
	\item $f(x) = \displaystyle \frac{1}{1 - x}$.               
\end{enumerate}
Para cada desarrollo en serie se pide:    



\begin{enumerate}
	\item Calcular el desarrollo para un valor de $M$ genérico ($M=1, 2, 3, 4, 5, \hdots$) mediante un bucle desde $ k=0$ 
	hasta $ k = M $. 
	\item Implementar un desarrollo para infinitos términos que para de sumar cuando se alcance la máxima precisión. 
	Realizar la implementación mediante 
	una sentencia de control  \verb|while|.   
	\item Comparar con las funciones intrínsecas correspondientes.     
	% \item  Dibujar la función a aproximar junto con las param�tricas que constituyen los desarrollos anteriores con $ M=1, 2, 3, 4, 5. $
	\item Discutir los resultados obtenidos.
\end{enumerate}


















\newpage 

%***********************************************
\item   Vectores y matrices. 

\noindent Sean los vectores $V, W \in \mathbb{R}^N$ y la matriz de Vandermode $A \in 
{\cal M}_{N \times N} (\mathbb{R})$: 
$$
\{ v_i =  1/i^2, \ \  i=1, \ldots N \} \qquad  \{  w_i = (-1)^{i+1}/(2i+1), \ \   i=1, \ldots N  \} 
$$
$$
\{ a_{ij} = (i/N)^{j-1}, \ \ i=1, \ldots N, \ \  j=1, \ldots N \}. 
$$
Escribir un programa que realice las operaciones siguientes y que muestre los
resultados por pantalla: 
\begin{enumerate}
	\item Sumar todas las componentes del vector $V$. 
	\item Sumar todas las componentes de la matriz $A$.   
	\item Sumar las componentes del vector $V$ mayores que cero.
	\item Sumar las componentes de la matriz $A$ menores que cero.   
	\item Producto escalar de los vectores $V$ y $W$.   
	\item Producto escalar del vector $V$ y la columna $N$ de la matriz $A$. 
	\item Multiplicar la matriz $A$ por el vector $V$.
	\item Matriz traspuesta de la matriz $A$   
	\item Obtener el valor máximo de la matriz $A$ y su posición.
\end{enumerate}


\item Dada la matriz  de Vandermonde $A_M \in {\cal M}_{M \times M} (\mathbb{R})$ del hito anterior,
calcular las siguientes operaciones: 

\begin{enumerate}	
	\item Suma de las trazas de $ A_M $ para matrices de diferente dimensión. 
	$$  S_1 = \sum_{M=1} ^{10} trace(A_M) $$ 
	\item  Suma de las trazas del cuadrado de $ A_M $ para matrices de diferente dimensión. 
	$$  S_2 = \sum_{M=1} ^{5} trace(A_M^2) $$ 
	\item Traza de la suma de potencias de $ A_M $  con $ M=8 $.
	$$  S_3 = trace \left( \sum_{k=0} ^{5} A_8^k  \right) $$
\end{enumerate}  
 
Implementar la operaciones anteriores mediante un módulo llamado \textbf{Algebra} 
que contenga la función  \textbf{trace} que calcula la traza de una matriz, la 
función  \textbf{Vandermonde} que crea la matriz  $ A_M \in { \cal M } _{ M 
\times M } ( \mathbb{R} )$ 
 y la función \textbf{power} que permite calcular la potencia k--ésima de 
 cualquier matriz cuadrada. 



%***********************************************
\item  Sentencias de entrada y salida.
\vspace{0.5cm}    
Crear los ficheros de datos con nombres  \verb|data1.txt| e 
\verb|data2.txt| con la información siguiente:
\vspace{1cm}
Contenido del fichero de entrada \verb|data1.txt| :
\begin{verbatim}
VARIABLES=t, x, y, z, u
1.2,      3.4,     6.2,    -14.0,    0.1
-25.2,    -8.6,    5.1,     9.9,    17.0
-1.0,     -2.0,    -5.4,    -8.6,    0.0
3.14,     -11.9,   -7.0,    -12.1,   9.2
6.66,    5.32,    0.001,   0.2,     0.001
\end{verbatim}
\vspace{0.1cm}
Contenido del fichero de entrada \verb|data2.txt| :
\begin{verbatim}
VARIABLES=t, x1, x2, x3, x4, x5, x6
1.2,      3.4,     6.2,    -14.0,    0.1,      4.89,   7.54
4-25.2,    -8.6,    5.1,    12.0,     9.9,      12.24,  17.0
0.0,      34.5,    -1.0,   -2.0,     -43.04,   -8.6,    0.0
3.14,     -11.9,   71.0,   7.0,     17.0,      -12.1,   9.2
 6.66,     5.32,    0.001,   0.2,     0.001,    0.008,   -0.027
54.0,     77.1,    -9.002,  -13.2,   0.017,    65.53,   -0.021
 23.04,    -51.98,  -34.2,   9.99,    5.34,     8.87,    3.22 
\end{verbatim}
\vspace{0.5cm}
Escribir una función \verb|load_matrix| que permita crear una matriz A con las dimensiones 
del fichero de datos y que almacene los valores de dicho fichero mediante 
 asignaciones del tipo: 

 \hspace{1cm} \verb|A = load_matrix('data1.txt')| 
 \hspace{1cm}  \verb|A = load_matrix('data2.txt')|

La función   \verb|load_matrix| deberá permitir la carga de datos con cualquier formato. 
Una vez cargados los valores de la matriz $ A $, crear dos vectores $ U$ y $V$  que contengan la primera y segunda columnas de la matriz A. 
     




	
\item Escribe una función de nombre collatz $(\mathrm{N})$ 
que reciba como argumento un numero natural $\mathrm{N}$ y 
devuelva una lista con la sucesión $\left\{a_{1}, a_{2}, \ldots, a_{i}, \ldots\right\}$ 
generada a partir de $a_{1}=$ $N$ usando la fórmula:
	Con:
	$$
	a_{i+1}=f\left(a_{i}\right)
	$$
	$$
	f(x)=\left\{\begin{array}{cl}
		\frac{x}{2} & \text { si x es par } \\
		3 x+1 & \text { si x es impar }
	\end{array}\right.
	$$
	Devolverá la lista cuando se alcance $a_{i}=1$ \\
	Ej collatz $(7)=[7,22,11,34,17,52,26,13,40,20,10,5,16,8,4,2,1]$
	
\end{enumerate} 

\end{document}