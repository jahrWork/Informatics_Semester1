%
\documentclass[12pt,spanish]{article}
   


\usepackage[spanish]{babel}
\usepackage[utf8]{inputenc}

\usepackage[fleqn]{amsmath}
\usepackage{nccmath}
\usepackage{amsmath}
\usepackage{nccmath}

\usepackage{latexsym}  %\mathbb{R}

%**************

\usepackage{graphicx}
\usepackage{bezier}
\usepackage[spanish]{babel}
\usepackage{makeidx}
\usepackage{latexsym}
% acentos y cosas varias
\usepackage{epstopdf}
\usepackage{amssymb}
\usepackage{curves}
\usepackage{rotating}
\usepackage{amsmath}
\usepackage{epic}


%\usepackage{HVDASHLN}
%\usepackage{epsfig}


\usepackage[utf8]{inputenc}
\usepackage[T1]{fontenc}
\usepackage{lmodern} % load a font with all the characters


\usepackage{hyperref} 
\usepackage{xcolor}

%**************



%*************************************************************************************************************
%***  Formato del documento                                                                                ***
%*************************************************************************************************************

\textwidth       16 cm
\textheight      24 cm

\oddsidemargin    0 cm
\evensidemargin   0 cm

\topmargin        -2 cm


\pagestyle{empty}

%\parskip = 1.5 \baselineskip


%*************************************************************************************************************
%***  Comandos para distintos conjuntos de numeros                                                         ***
%*************************************************************************************************************

\newcommand{\RR}{{\sf I}\hspace*{-1.5pt}{\sf R}}
\newcommand{\NN}{{\rm I}\hspace*{-1.5pt}{\rm N}}
\newcommand{\PP}{{\sf I}\hspace*{-1.5pt}{\sf P}}
\newcommand{\QQ}{\hspace*{2pt}{\sf I}\hspace*{-5pt}{\sf Q}}
\newcommand{\ZZ}{{\sf Z}\hspace*{-4.5pt}{\sf Z}}
\newcommand{\CC}{\hspace*{2.5pt}{\sf I}\hspace*{-5pt}{\rm C}}


%*************************************************************************************************************
%***  Comandos para subindices y superindices                                                              ***
%*************************************************************************************************************
 
\newcommand{\Ri}{_{_R}}  
\newcommand{\Le}{_{_L}}

      

%*************************************************************************************************************
%***  Inicio de documento                                                                                  ***
%*************************************************************************************************************



\begin{document}

\begin{center}
	{\bf \Large Asignatura : 
	Informática}                                               
	\\ \vspace{1cm}
	{\bf \Large Grado en Ingeniería Aeroespacial - 
	ETSIAE}                                            \\
%	{\bf \Large Curso : 
%	2021-2022}                                                                  
	            %   \\
\end{center}

\vspace{1cm}

\noindent
{\bf \Large Subjects for the P3 assignment: 
}                                                                               
     
\vspace{0.5cm}

\begin{enumerate}

%\item Catastrophic cancellation in numerical computations. Application to 
%specific problems such as roots of a second grade equation. 

%\item Internal representation of numbers. Standard IEEE. IEEE exceptions. 
%How to detect problem: underflow, overflow. 

%\item Ill posed problems and stability. 

%\item History of Fortran language with code examples. From Hidden Figures to 
%our days. 

\item Emergent behavior: flocking, ants, society, fire evacuation, viral infection. 
\item Games: SUDOKU, ..,

%\item Functional programming. John Backus contributions. Declarative 
%(functional paradigm) versus imperative (von Neumann style, "word at a time" 
%paradigm). 

%Lazy evaluation. Expressiveness. Data abstraction. Map, filter and 
%reduce. 

%\item Determinant of a matrix using the Laplace expansion by means of a 
%recursive function:  
%$$ 
%   \det(A) = \sum _{j=1} ^N (-1)^{i+j} a_{ij} M_{ij}, 
%$$
%where $ M_{ij} $ stands for the determinant of a $ (N-1) \times (N-1) $ matrix 
%by removing ith row and jth column. 
%
%\item Application to visualize Taylor expansion convergence at different 
%points. Implement a friendly graphical user interface to analyze and to 
%understand convergence pitfalls. 

\item Music generation with the computer. 

\item Generation of musical notes. Harmonic composition of different instruments. Visualization and sound effect.
Equal temperament versus Pythagorean tuning. 

\item Music perception. Missing fundamental. Quality of chords. Consonance and Dissonance. 
Reverberation versus echoing. 

\item Numerical integration of Kepler orbits by means of the Euler method. 
 
 \item Fractals: Mandelbrot set, Koch snowflake, Sierpiński triangle,... 
 
\end{enumerate}
The program of the  P3 assignment can be implemented in Python, FORTRAN, C++ or JavaScript. 


\vspace{0.5cm} 
\noindent
{\bf \Large General contents of P3 assignment: 
}                                                                               
     
\begin{enumerate}
 \setlength{\itemsep}{-0.1cm}
\item Objective. 
\item State of art. 
\item Explanation through programming codes. 
\item Conclusions. 
\item References. 
\end{enumerate}









\end{document} 
 
 
 
 
 
 
 
 
 
 
 
 
 
 
 
 
 
 
 
 
 
 
 
 
 
 
 
 
 
 
 
 
 
 
 
 
\newpage


%***********************************************
    \item {\bf Hito 5 : }   Funciones y gráficas. 
 
 \begin{enumerate}    
 \item Escribir un programa que dibuje las siguientes funciones: 
 
 
 
 \begin{fleqn}
  \[ f(x)  = \sqrt{ 1 - ( |x| -1 )^2 } \]
  
  \[ f(x) = \arccos(1 - | x | )  + \pi \]
  \[ f(x) =  \frac{ \cos( 70 x ) }{ (1 + x^2 )^2  }\]
 \[ f(x) = \sqrt{ \cos(x) }  \]
  \[ f(x) =  8 x \sin( x^2 + 3)  \]
 \[ f(x) = \prod_{k=0} ^M ( x - k/M )  \]
 \end{fleqn}
 
\item Dibujar el lugar geométrico de las raíces de: 

\[ x^2 + 2 \zeta \omega_0 x + \omega_0 ^2 = 0,\]

con $ \omega_0 = 1 $ para todo $ \zeta \ge 0$. 
 
\end{enumerate}



%***********************************************
    \item {\bf Hito 6 :}  Aproximar  mediante un desarrollo en serie de potencias de la forma
    \[  f(x) = \sum_{k=0} ^M a_k \  x^k, \qquad a_k = \frac{  f^{(k)} (0)  }{ k! }  \]  
    las  siguientes funciones: 
   
    \begin{fleqn}
  \[   f(x)  = \cos(x)  \]
    \[ f(x) = \cosh(x) \]
 \end{fleqn}
    
 
 
 
 \begin{enumerate}
   \item Calcular los desarrollos anteriores para un valor de $ M $ genérico.  
   \item  Dibujar la función a aproximar junto con las parmétricas que constituyen los desarrollos anteriores con $ M=1, 2, 3, 4, 5. $
   \item Discutir los resultados obtenidos.
 \end{enumerate}
 
 

%***********************************************
    \item {\bf Hito 7 :}   Ceros de funciones. 
    
  Dada una función genérica $ f(x) $ escribir una programa que permita calcular los ceros mediante: 
 
 
 \begin{enumerate}
   \item Algoritmo de la bisección. 
   \item Algoritmo de Newton. 
   \item Hacer aplicación a una función ejemplo y dibujar el resultado obtenido. 
 \end{enumerate}
 
\end{itemize}

\end{document}        
    
   
 
